\documentclass{article}
\usepackage[english,russian]{babel} % на каких языках написан текст. В идеале нужно для переносов, которых строго говоря по стандарту МИРЭА в отчётах и работах НЕТ (хотя разница до и после отключения переносов ужасная)
\usepackage{fontspec}
\setmainfont{Times New Roman}
\setmonofont{Consolas}
\begin{document}
\section{Что это такое}
Данный набор шаблонов и скриптов предназначен для упрощения создания отчётов по систенме \LaTeX  при сохранении соответствия внутреннему стандарту МИРЭА. Работает на Linux, скорее всего так же будет работать на иных юникс-подобных ОС (macOS, *BSD), возможно после серьёзной доснастройки также будет работать и на Windows.

Помимо собственно адаптации стилей автоматизировано построение титульного листа. В первых четырёх строках пишутся:
\begin{enumerate}
\item тип работы (допустимы сокращения: кр --- курсовая работа, пр --- отчёт по практической работе, лр --- отчёт по лабораторной работе, и наконец для любителей писать конспекты есть особый тип: лк --- отчёт по лекционной (конспективной) работе.
\item тема работы (без кавычек)
\item дисциплина (можно сокращенно, если настроено раскрытие сокращений предметов - об этом ниже) % TODO здесь должна быть ссылка на соответствующий раздел
\item преподаватель (можно одну фамилию - но здесь необходима настройка преподавателей) % ТОDO здесь должна быть ссылка на соответствующий раздел
\end{enumerate}

\section{Требования (какой софт нужно предустановить)}

Для компиляции отчётов МИРЭА рекомендуются системы XeLaTex, так как именно они осуществляют полную поддержку Unicode, ныне господствующего во всех современных юниксах, а также использование системных шрифтов (в том числе настоящего Times New Roman); шаблоны отчётов и стилей выстроены под это семейство \LaTeX.

Для корректной работы потребуется установка ttf-шрифтов Times New Roman и Consolas - это необходимо для реализации внутреннего стандарта МИРЭА. Вопрос установки шрифтов в различных системах рассматривать не буду (поскольку с точки зрения закона об авторском праве Вы не должны нарушить копирайт корпорации Майкрософт), однако скажу, что чисто технически скрипту нужно лишь, чтобы ttf-файлы со шрифтами Times New Roman и Consolas находились в системной директории шрифтов (для Linux это, как правило, \tt/usr/share/fonts/\rm или конкретнее \tt/usr/share/fonts/TTF\rm).

\section{Особенности}
Для форматирования иллюстраций используется стандартное для LaTeX окружение \verb|\figure|, для формул стандартное \verb|$ ... $| (внутритекстовое) и \verb|$$ ... $$| (внетекстовое).
Для пока не реализованных (см.ниже) % ссылка! желательно кликабельная!
таблиц \it будет\rm~использоваться cтандартное окружение table. \bf А вот для листингов используется окружение lstlisting из пакета listings.\rm
\section{Планы на будущее (TODO) - пока не реализованные функции}
\begin{enumerate}
	\item Оформление таблиц по стандарту МИРЭА
	\item Нумерация формул по стандарту МИРЭА
	\item \verb|\смфор| и \verb|\смкод|
	\item Список источников по ГОСТ
	\item (возможно что сделаю) портирование на Windows
\end{enumerate}

Также к отчётам есть требования, которые сложно автоматизировать и о которых скорее всего придётся помнить всегда (пожалуйста, если у вас есть идеи как это исправить, предлагайте варианты).
\begin{enumerate}
\item В русском тексте следует использовать <<кавычки-ёлочки>> (\verb|<<кавычки-ёлочки>>|, в английском ,,кавычки-лапки`` (\verb|,,кавычки-лапки``|), а в листингах кода сохранять код неизменным (то есть скорее всего будут задействованы \verb|'прямые одиночные'| или \verb|"прямые двойные"| кавычки-лапки).
\item Тире записывается как длинное тире --- (тремя дефисами \verb|---|, а среднее тире -- (в латехе это два дефиса \verb|--|) можно использовать только в формулах как знак минус (но тут уже сам латех сам позаботится об отрисовке правильного знака)
\item Правильно: <<представлено на Рисунке 2.4>>; <<находится в Приложении А>>; <<данную закономерность мы можем наблюдать в Таблице 3.5>>. \verb|\смрис| и \verb|\смкод| дают лишь стандартную запись в именительном падеже и в скобках (Рисунок 2.4)
\end{enumerate}

\end{document}
